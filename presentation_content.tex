\documentclass{article}
\usepackage[utf8]{inputenc}
\usepackage{geometry}
\usepackage{graphicx}
\usepackage{hyperref}

\geometry{a4paper, margin=1in}

\title{\textbf{KAVACH-AI: Multimodal Deepfake Detection \& Forensic Analysis Platform}}
\author{Secure Key: admin@kavach.ai}
\date{\today}

\begin{document}

\maketitle

\section{Abstract}
KAVACH-AI is an enterprise-grade forensic platform designed to combat the proliferation of AI-generated media (deepfakes). As synthetic media becomes increasingly sophisticated, traditional single-modal detection methods fall short. KAVACH-AI employs a novel \textbf{Multi-Modal Fusion Engine} that simultaneously analyzes video streams for spatial artifacts (using MesoNet), audio tracks for spectral inconsistencies (using Mel-Spectrogram analysis), and temporal sequences for unnatural movement patterns (using LSTM networks). This triangulated approach significantly reduces false positives and provides high-confidence verdicts. The system is built on a scalable microservices architecture, featuring real-time RTSP stream ingestion, blockchain-ready forensic chain of custody, and a comprehensive analyst dashboard, making it a robust solution for law enforcement and digital forensic laboratories.

\section{Objective}
The primary objectives of the KAVACH-AI platform are:
\begin{itemize}
    \item \textbf{High-Confidence Detection:} To detect deepfakes with superior accuracy by fusing insights from visual, audio, and temporal analysis models.
    \item \textbf{Real-Time Surveillance:} To provide live threat detection capabilities capable of processing RTSP video streams in real-time.
    \item \textbf{Forensic Integrity:} To ensure the admissibility of evidence through a blockchain-ready chain of custody, utilizing cryptographic hashing and immutable audit logs.
    \item \textbf{Scalability \& Accessibility:} To deliver a modular, Dockerized solution that can be easily deployed in various environments without reliance on external paid APIs.
\end{itemize}

\section{Existing System}
Current deepfake detection solutions primarily focus on single-modal analysis:
\begin{enumerate}
    \item \textbf{Frame-by-Frame CNNs:} These systems analyze individual video frames for visual artifacts. While effective against low-quality deepfakes, they often fail to detect temporally consistent high-quality generations and struggle with video compression artifacts.
    \item \textbf{Audio-Only Analysis:} Tools that focus solely on synthetic voice detection. These miss visual cues and fail when audio is real but the video is manipulated (e.g., lip-sync deepfakes).
    \item \textbf{Lack of Real-Time Processing:} Most existing tools are designed for offline file analysis, making them unsuitable for live monitoring of broadcast or surveillance feeds.
    \item \textbf{Opague Verification:} Detailed forensic evidence and explainability are often missing, providing only a binary "Real/Fake" output without supporting data for legal contexts.
\end{enumerate}
\textbf{Limitations:} High false positive rates, inability to handle multi-modal manipulation, lack of real-time capability, and insufficient forensic audit trails.

\section{Proposed System}
KAVACH-AI introduces a holistic, multi-layered approach to deepfake detection:
\begin{itemize}
    \item \textbf{Multi-Modal Fusion Engine:} A sophisticated weighted voting ensemble that combines:
    \begin{itemize}
        \item \textbf{Spatial Analysis:} MesoNet/EfficientNet models to detect visual artifacts within frames.
        \item \textbf{Spectral Analysis:} Audio analysis using Mel-Spectrograms to identify synthetic voice signatures.
        \item \textbf{Temporal Analysis:} LSTM networks to detect unnatural motion and temporal inconsistencies across frame sequences.
    \end{itemize}
    \item \textbf{Real-Time Processing Pipeline:} An asynchronous, non-blocking ingestion engine capable of handling live RTSP and YouTube Live streams with adaptive sampling.
    \item \textbf{Forensic Dashboard:} An interactive React-based frontend providing frame-by-frame anomaly visualization, confidence scores, and detailed modality breakdowns.
    \item \textbf{Immutable Evidence Chain:} Usage of Merkle trees and cryptographic hashing to maintain a tamper-proof chain of custody for all analyzed media.
    \item \textbf{Privacy-First Design:} All processing is performed locally with no dependency on external APIs, ensuring data privacy and security.
\end{itemize}

\section{System Architecture}
The KAVACH-AI platform follows a modular microservices architecture, ensuring scalability, maintainability, and fault tolerance.

\subsection{Architectural Layers}

\begin{enumerate}
    \item \textbf{Client Layer:}
    \begin{itemize}
        \item \textbf{Frontend:} Built with \textbf{React.js} and \textbf{Vite}, delivering a responsive dashboard for monitoring and analysis.
        \item \textbf{Data Visualization:} Utilizes interactive charts to display confidence trends and forensic analytics.
    \end{itemize}

    \item \textbf{Application Layer:}
    \begin{itemize}
        \item \textbf{API Gateway:} A high-performance \textbf{FastAPI} (Python) backend handling RESTful requests, authentication, and stream management.
        \item \textbf{Authentication:} secure JWT-based authentication system handling user sessions and role-based access control.
        \item \textbf{Task Queue:} \textbf{Celery} workers enabling asynchronous background processing for heavy AI inference tasks.
    \end{itemize}

    \item \textbf{Data Layer:}
    \begin{itemize}
        \item \textbf{Database:} \textbf{PostgreSQL} (or SQLite for local dev) stores metadata, user profiles, and detection logs.
        \item \textbf{Cache \& Message Broker:} \textbf{Redis} manages the task queue, caches frequent queries, and handles real-time Pub/Sub for WebSocket updates.
        \item \textbf{Evidence Storage:} Secure local storage for video artifacts and generated forensic reports.
    \end{itemize}

    \item \textbf{AI Core Layer:}
    \begin{itemize}
        \item \textbf{Inference Engines:} Dedicated modules for Video (MesoNet), Audio (Spectrogram execution), and Temporal (LSTM) models using \textbf{ONNX Runtime} and \textbf{PyTorch}.
        \item \textbf{Fusion Module:} Aggregates individual model outputs to generate a final weighted confidence score.
    \end{itemize}
\end{enumerate}

\end{document}
